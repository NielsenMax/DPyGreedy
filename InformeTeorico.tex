\documentclass[10pt]{article}
\usepackage[cp1252]{inputenc}
\usepackage{graphicx}
\usepackage{vmargin}
\usepackage{fancyhdr}

\setpapersize{A4}
\setmargins{2cm}		%margen izquierdo
{3cm}				%margen superior
{17.5cm}				%anchura del texto
{23.3cm}				%altura del texto
{10pt}				%altura de los encabezados
{1cm}				%espacio entre el texto y los encabezados
{10pt}				%altura del pie de pagina
{0.5cm}				%espacio entre el texto y el pie de pagina

\begin{document}
%%%%%%%%%%%%%%%Pie de Pagina y Encabezado
\pagestyle{fancy}
\renewcommand{\headrulewidth}{0.4pt}
\renewcommand{\footrulewidth}{0.4pt}
\fancyhead[LO,LE]{Algoritmos y Estructuras de Datos Avanzadas}
\fancyfoot[C]{15/10/2019}
\fancyhead[R]{Nielsen Maximiliano, Uliassi Manuel}
\fancyfoot[L]{6to Informatica}
\fancyfoot[FR] {P\'agina \thepage}
%%%%%%%%%%%%%%%Pie de Pagina y Encabezado

%%%%%%%%%%%%%%%%%%%%%PORTADA
\begin{titlepage}
\null
\vspace{4cm}
\centering
{\Huge Programaci\'on Din\'amica\\ y Greedy}

\vspace{0.2cm}

{\LARGE Algoritmos y Estructuras de Datos Avanzadas}

\vspace{0.2cm}

{\large Informe Te\'orico}

\vspace{1.5cm}

{\LARGE Nielsen Maximiliano - Uliassi Manuel}

\vspace{0.15cm}

{\large Docente: Juan Manuel Rabasedas}

\vspace{0.15cm}

{\large 6to Inform\'atica}

\vspace{0.15cm}

{\large Instituto Politecnico Superior Gral. San Martin}
\vfill

\end{titlepage}
%%%%%%%%%%%%%%%%%%%%%PORTADA
\begin{center}
\section*{\LARGE Informe Te\'orico}
\end{center}
\subsection*{Objetivo}
El objetivo de este Informe Teorico es explicar y complementar con teoria la resoluci\'on de los siguientes problemas de Programaci\'on Din\'amica y Algoritmos Greedy asignados.
\subsection*{Problema de Programaci\'on Din\'amica}
\end{document}
